\documentclass[prd,aps,nofootinbib,floatfix,10pt]{revtex4}
%\usepackage{amsmath,graphicx,color,epsfig}
%dsfont, mathtools
\usepackage{amsmath,graphicx,epsfig,amssymb}
\usepackage{inputenc}
\usepackage[usenames]{color}
\usepackage{ulem} %% for strike-through
\usepackage{bigstrut}
\usepackage{slashed}
\usepackage{multirow}
\usepackage{subfigure}
\usepackage{cancel}
\renewcommand{\baselinestretch}{1.3}
\allowdisplaybreaks

\newcommand{\blue}[1]{{\color{blue} #1}}
\newcommand{\red}{\textcolor{red}}



%%%%%%%%%%%%%%%%%%%%%%%%%%%%%%%%%%%%%%%%
\begin{document}

\title{Peskin Solutions: Chapter 9}
\author{Jinchen}
\maketitle


\section{How to use the functional method to get propogator.}

\red{According to (9.34), generating functional $Z[J] = \int \mathcal{D}\phi [\exp(i\int d^4 x \mathcal{L}) \cdot \exp(i\int d^4 x J(x) \phi(x))]$. Then change the variable to get $Z[J] = \int \mathcal{D}\phi' [\exp(i\int d^4 x \mathcal{L'}) \cdot \exp(-\frac{i}{2} \int d^4 x J \hat{O}^{-1} J )]$. Here the current term is irrelavant to $\phi'$, so $Z[J] = Z_0 \cdot \exp(-\frac{i}{2} \int d^4 x J \hat{O}^{-1} J ) $, and the functional derivatives will be applied on the current term. }

\section{Problem 9.1}

\noindent (a)

\[
    \mathcal{L} = - \frac{1}{4} F^2_{\mu \nu} + (\partial_{\mu} \phi^* - i e A_{\mu} \phi^*)(\partial^{\mu} \phi + i e A^{\mu} \phi) - m^2 \phi^* \phi = \mathcal{L}_{A} + \mathcal{L}_{\phi} + \mathcal{L}_{I}
\]

The $\mathcal{L}_{A}$ is just free E-M field, so the propogator is the propogator of photon. 

The $\mathcal{L}_{\phi} = \partial_{\mu} \phi^* \partial^{\mu} \phi - m^2 \phi^* \phi = \partial_{\mu} (\phi^* \partial^{\mu} \phi) - \phi^* \partial_{\mu} (\partial^{\mu} \phi) - m^2 \phi^* \phi$, because the differential term in the Lagrangian density makes no difference, we got $\mathcal{L}_{\phi} = - \phi^* \partial_{\mu} (\partial^{\mu} \phi) - m^2 \phi^* \phi = \phi^* (- \partial^2 - m^2) \phi = \phi^* \hat{T} \phi$.

With generating functional method, we have $\mathcal{L}_{\phi} + \eta^* \phi + \phi^* \eta$ in the $Z[J]$, then do a shift $\phi \to \phi' = \phi + \hat{T}^{-1} \eta$, we got $\mathcal{L}_{\phi} + \eta^* \phi + \phi^* \eta = \mathcal{L}_{\phi'} - \eta^* \hat{T}^{-1} \eta$. If $G$ is the Green function of $\hat{T}$, then $\mathcal{L}_{\phi'} - \eta^* \hat{T}^{-1} \eta = \mathcal{L}_{\phi'} - \eta^* (iG * \eta)$,

\[ Z[\eta, \eta^*] = Z_0 \cdot \exp[-i\int d^4 x d^4 y \ \eta^*(x) i G(x-y) \eta(y) ] \]

\[ \text{prop} = -\frac{\delta}{\delta \eta^*}\frac{\delta}{\delta \eta} \exp[-i\int d^4 x d^4 y \ \eta^*(x) i G(x-y) \eta(y) ] = -G \]

After two functional derivatives, we will find the propogator is exactly the $-G$.

So the propogator of $\phi$ and $\phi^*$ is $\frac{i}{p^2 - m^2 + i \epsilon}$. (How to calculate the Green function of $\hat{T}$ - \blue{Check Eq.(2.57) in Peskin})

\[ \hat{T}^{-1} \eta(x) = i \int d^4 y \ G(x-y) \eta(y) \]

\[ \hat{T} G(x-y)  = (-\partial^2-m^2) G(x-y) = -i \delta(x-y) \]

FT to get,

\[ (p^2 - m^2) \tilde{G}(p) = -i \]

\[ -\tilde{G}(p) = \frac{i}{p^2 - m^2} \]

Then comes to vertices, $\mathcal{H}_{I} = - \mathcal{L}_I$ (\blue{P. 289 in Peskin}), theoretically we should check \blue{Eq.(4.31)} and do the contraction to get Feynman rules, but here we can just look at $\exp[i \int \mathcal{L}_I]$, here $\mathcal{L}_I = i e g^{\mu \nu} (\partial_{\mu} \phi^* A_{\nu} \phi - A_{\mu} \phi^* \partial_{\nu} \phi) + e^2 g^{\mu \nu} A_{\mu} \phi^* A_{\nu} \phi$, then $i \mathcal{L}_I = -i e g^{\mu \nu} (-i \partial_{\mu} \phi^* A_{\nu} \phi + A_{\mu} \phi^* i \partial_{\nu} \phi) + i e^2 g^{\mu \nu} A_{\mu} \phi^* A_{\nu} \phi$.

There are three terms, let's throw those fields away and turn $i \partial \phi$ to $p_{\phi} \phi$, $-i \partial \phi^*$ to $p_{\phi^*} \phi^*$, here $p$'s are along particle/anti-particle lines, besides, the third term has two $A$ fields, which are commutative, so there should be a factor $2$ for the $AA\phi^*\phi$ vertex.

So, 

\[For\ \phi^* A \phi: -i e (p + p')^{\mu}\]

\[For\ AA\phi^* \phi: 2 i e^2 g^{\mu \nu}\]


Theoretically, 

\[ <\phi \phi^* | S | \gamma> = <\phi \phi^* | T \int d^4 x i \mathcal{L}_I | \gamma> = <\phi \phi^* | T \int d^4 x (-i e) g^{\mu \nu} (-i \partial_{\mu} \phi^* A_{\nu} \phi + A_{\mu} \phi^* i \partial_{\nu} \phi) | \gamma> \]

and 

\[ <\phi \phi^* | S | \gamma \gamma> = <\phi \phi^* | T \int d^4 x i \mathcal{L}_I | \gamma \gamma> = <\phi \phi^* | T \int d^4 x (i e^2) g^{\mu \nu} A_{\mu} \phi^* A_{\nu} \phi | \gamma \gamma> \]

give the Feynman rules of two kinds of vertex with contractions.


~\\
\noindent (b)

With \blue{Eq.(4.84)}, $m_e$ is ignored, then,

\[(\frac{d \sigma}{d \Omega})_{c.m.} = \frac{|\vec{p}_{\phi}|}{32 (2\pi)^2 E_{e}^2 \cdot 2 E_e } \frac{1}{4} \Sigma |\mathcal{M}(ee \to \phi^*\phi)|^2\]

The outlines of $\phi$ and $\phi^*$ are $1$, the Feynman diagram looks similar to the diagram in \blue{P.131}.

\[ i \mathcal{M} = (-i e)^2 \bar{v}(k_2) \gamma^{\mu} u(k_1) \frac{-i g_{\mu \nu}}{s + i \epsilon} (p_1 - p_2)^{\nu} = i e^2 \bar{v}(k_2) (\cancel{p_1} - \cancel{p_2}) u(k_1) \frac{1}{s + i \epsilon}\]

\[\frac{1}{4} \sum_{\text{spin}} |\mathcal{M}|^2 = \sum_{\text{spin}} \frac{e^4}{4 s^2} \bar{v}(k_2) (\cancel{p_1} - \cancel{p_2}) u(k_1) \bar{u}(k_1) (\cancel{p_1} - \cancel{p_2}) v(k_2) \]

\[ = \sum_{\text{spin}} \frac{e^4}{4 s^2} \text{tr}( v(k_2) \bar{v}(k_2) (\cancel{p_1} - \cancel{p_2}) u(k_1) \bar{u}(k_1) (\cancel{p_1} - \cancel{p_2}) ) \]

\[ = \frac{e^4}{4 s^2} \text{tr}( \cancel{k_2} (\cancel{p_1} - \cancel{p_2}) \cancel{k_1} (\cancel{p_1} - \cancel{p_2}) ) \]

\[ = \frac{e^{4}}{4 s^{2}}\left[8\left(k_{1} \cdot p_{1}-k_{1} \cdot p_{2}\right)\left(k_{2} \cdot p_{1}-k_{2} \cdot p_{2}\right)-4\left(k_{1} \cdot k_{2}\right)\left(p_{1}-p_{2}\right)^{2}\right] \]

Choose a specific frame, 

\[ k_{1}=(E, 0,0, E), \ \  p_{1}=(E, p \sin \theta, 0, p \cos \theta) \]

\[k_{2}=(E, 0,0,-E), \ \  p_{2}=(E,-p \sin \theta, 0,-p \cos \theta)\]

\[ \frac{1}{4} \sum_{\text{spin}} |\mathcal{M}|^2 = \frac{e^{4} p^{2}}{2 E^{2}} \sin ^{2} \theta \]

$ee \to \mu \mu$ is \blue{Eq.(5.11)}

So,

\[
    \left(\frac{\mathrm{d} \sigma}{\mathrm{d} \Omega}\right)_{\mathrm{c.m.}}=\frac{1}{2(2 E)^{2}} \frac{p}{8(2 \pi)^{2} E}\left(\frac{1}{4} \sum|\mathcal{M}|^{2}\right)=\frac{\alpha^{2}}{8 s}\left(1-\frac{m^{2}}{E^{2}}\right)^{3 / 2} \sin ^{2} \theta 
\]

~\\
\noindent (c)

Two diagrams because there are two kinds of vertex which are listed in (a). Because the minus signs in the $\mathcal{L}_I$ are all absorbed in vertex, and there is no Fermion field, the sign between two vertices is $+$. That's why the two diagrams should be added.


\[ i \Pi^{\mu \nu}_1 = e^2 \int \frac{d^4 k}{(2\pi)^4} (2k + q)^{\mu} \frac{1}{k^2 - m^2 + i \epsilon} (2k+q)^{\nu} \frac{1}{(k+q)^2 - m^2 + i \epsilon} \]

\[ i \Pi^{\mu \nu}_2 = - 2 e^2 g^{\mu \nu} \int \frac{d^4 k}{(2 \pi)^4} \frac{1}{(k+q)^2 - m^2 + i \epsilon} \]

add togeter, get

\[ i \Pi^{\mu \nu} = - e^2 \int \frac{d^4 k}{(2\pi)^4} \frac{2 g^{\mu \nu} (k^2 - m^2) - (2k + q)^{\mu} (2k+q)^{\nu}}{(k^2 - m^2)((k+q)^2 - m^2)} \]

\[ \frac{1}{(k^2 - m^2)((k+q)^2 - m^2)} = \int^1_0 dx \frac{1}{[(k + (1-x)q)^2 + xq^2 -x^2 q^2 - m^2]^2} \]

change the variable, $l = k + (1-x)q$, with \blue{Eq.(6.45)},

\[ \text{numerator} = g^{\mu \nu} l^2 + 2 g^{\mu \nu} (1-x)^2 q^2 - 2 g^{\mu \nu} m^2 - (2x-1)^2 q^\mu q^\nu \]

do the Wick rotation, $l^0 = i l^0_E$ and $l^i = l^i_E$, so we have $d^4 l = i d^4 l_E$ and $l^2 = - l^2_E$,

\[ i \Pi^{\mu \nu} = -i e^2 \int^1_0 dx \int \frac{d^4 l_E}{(2\pi)^4} \frac{-g^{\mu \nu} l_E^2 + 2 g^{\mu \nu} (1-x)^2 q^2 - 2 g^{\mu \nu} m^2 - (2x-1)^2 q^\mu q^\nu}{[l_E^2 + m^2 + x^2q^2 - xq^2]^2 } \]

\[= -i e^2 \int^1_0 dx \int \frac{d^4 l_E}{(2\pi)^4} [\frac{-g^{\mu \nu} l_E^2}{(l_E^2 + \Delta)^2} + \frac{2 g^{\mu \nu} (1-x)^2 q^2 - 2 g^{\mu \nu} m^2 - (2x-1)^2 q^\mu q^\nu}{(l_E^2 + \Delta)^2}] \]

use dimensional regularization, with \blue{Eq.(7.85) and Eq.(7.86)},

\[ i \Pi^{\mu \nu} = -i e^2 \int^1_0 dx [ (2 g^{\mu \nu} (1-x)^2 q^2 - 2 g^{\mu \nu} m^2 - (2x-1)^2 q^\mu q^\nu) \frac{1}{(4 \pi)^{d / 2}} \frac{\Gamma\left(2-\frac{d}{2}\right)}{\Gamma(2)}\left(\frac{1}{\Delta}\right)^{2-\frac{d}{2}} \]

\[ - g^{\mu \nu} \frac{1}{(4 \pi)^{d / 2}} \frac{d}{2} \frac{\Gamma\left(2-\frac{d}{2}-1\right)}{\Gamma(2)}\left(\frac{1}{\Delta}\right)^{2-\frac{d}{2}-1} ] \]

\[ = -i e^2 \int_0^1 dx \frac{1}{(4\pi)^{d/2}} (\frac{1}{\Delta})^{2-d/2} \Gamma(2-d/2) [ (2 g^{\mu \nu} (1-x)^2 q^2 - (2x-1)^2 q^\mu q^\nu) - g^{\mu \nu} \frac{d}{2-d} (x^2 q^2 - x q^2) ] \]

set $d = 4-\epsilon$ with $\epsilon \to 0$,

\[ i \Pi^{\mu \nu} = \frac{-i e^2}{(4\pi)^{2}} \int_0^1 dx  (\frac{\epsilon}{2} -\log\Delta - \gamma + \log(4\pi)) [ (g^{\mu \nu} (2x-2)(2x-1) q^2 - (2x-1)^2 q^\mu q^\nu) ] \]

Because $\int_0^1 dx  (\frac{2}{\epsilon} -\log\Delta - \gamma + \log(4\pi)) (2x-1) = \int_0^1 dx  \frac{2}{\epsilon} (2x-1) = 0$, we have

\[ i \Pi^{\mu \nu} = \frac{-i e^2}{(4\pi)^{2}} \int_0^1 dx  (\frac{\epsilon}{2} -\log\Delta - \gamma + \log(4\pi)) (2x-1)^2 [ (g^{\mu \nu} q^2 - q^\mu q^\nu) ] \]

with MS-bar scheme,

\[ \Pi(q^2) = \frac{-\alpha}{4\pi} \int_0^1 dx  (-\log\Delta ) (2x-1)^2 \]

If we adopt $-q^2 >> m^2$, 

\[ \Pi(q^2) = \frac{-\alpha}{4\pi} \int_0^1 dx  (-\log(x - x^2) - \log(-q^2) ) (2x-1)^2  \to \frac{-\alpha}{12 \pi} \log(-q^2)\]

while looking at Eq.(7.90), $\int_0^1 dx x(1-x) = \frac{1}{6}$, we know for $e+e-$ pair,

\[ \Pi(q^2) \to \frac{-\alpha}{3 \pi} \log(-q^2) \]

which is four times as our results.


\section{Problem 9.2}

\noindent (a)






\begin{thebibliography}{1}
	
\end{thebibliography}






\end{document}

